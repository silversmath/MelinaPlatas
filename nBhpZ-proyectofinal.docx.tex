\documentclass[12pt,a4paper]{article}
\usepackage[latin2]{inputenc}
\usepackage{graphicx}
\usepackage{ulem}
\usepackage{amsmath}
\begin{document}
\begin{figure}[h]
\centering
\includegraphics[width=15.92cm,height=14.02cm]{media/image1.eps}
\end{figure}




\begin{center}PROYECTO FINAL\end{center}



MELINA PLATAS SANCHEZ 

1740967

22-MAYO-2018

\textbf{INTRODUCCION:}

En este trabajo se realizara un programa en donde se explique la ruta 
m�s corta y en cuanto tiempo se realiza en 10 lugares que son municipios 
del estado de monterrey. No solamente usaremos Kruskal si no tambi�n 
ocuparemos de un programa llamado vecino m�s cercano. 

Y para este proyecto se necesita esencialmente conocer acerca de grafos 
para poder realizar las rutas m�s cortas.



\textbf{DEFINICIONES QUE OCUPAREMOS:}



\textbf{GRAFO: }

Conjunto de objetos llamados v�rtices o nodos unidos por enlaces 
llamados aristas o arcos, que permite relaciones binarias entre 
elementos de un conjunto.



\textbf{ALGORITMO: }

Es un conjunto prescrito de instrucciones o reglas bien definidas, 
ordenadas y finitas que permite llevar a cabo una actividad mediante 
pasos sucesivos.

\\
\textbf{ARBOL:}

Es una gr�fica el cual no existen ciclos.



\textbf{ALGORITMO KRUSKAL:}

Es un algoritmo para encontrar un �rbol de expansi�n m�nima en un grafo 
ponderado.







\textbf{ALGORITMO VECINO M�S CERCANO:}

Tambi�n llamado algoritmo voraz (greedy) permite al vigilante elegir la 
ciudad no visitada m�s cercana como pr�ximo movimiento. 



\begin{center}\textbf{PROGRAMAS EN PYTHON}\end{center}



\textbf{KRUSKAL}

\begin{figure}[h]
\centering
\includegraphics[width=16.72cm,height=14.12cm]{media/image2.eps}
\end{figure}


\textbf{VECINO M�S CERCANO}

\begin{figure}[h]
\centering
\includegraphics[width=16.14cm,height=10.02cm]{media/image3.eps}
\end{figure}




\end{document}
